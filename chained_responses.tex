\documentclass[11pt]{article}
\usepackage[a4paper,margin=1in]{geometry}
\usepackage{amsmath,amssymb,mathtools}
\usepackage{bm}
\usepackage{booktabs}
\usepackage{tabularx}
\usepackage[hidelinks]{hyperref}

\title{Mathematical Overview: Multiple Outcomes and Chained Races}
\author{AccumulatR notes}
\date{\today}

\newcommand{\Prob}{\mathbb{P}}
\newcommand{\E}{\mathbb{E}}
\newcommand{\1}{\mathbf{1}}

\begin{document}
\maketitle

\section{Goal}
This note summarizes what must be added mathematically to extend the current race framework to:
\begin{enumerate}
  \item multiple outcomes per trial (generation and likelihood), and
  \item chained races where later races start when earlier races finish, with possibly unobserved intermediate outcomes.
\end{enumerate}

\section{Current single-response race model}
\subsection{Accumulator-level timing}
For accumulator $i$, define a defective finishing time
\[
T_i=
\begin{cases}
o_i+t0_i+X_i, & \text{with probability } 1-q_i,\\
\infty, & \text{with probability } q_i,
\end{cases}
\]
where $X_i$ has base density/CDF $(g_i,G_i)$ on $(0,\infty)$.

Hence
\[
f_i(t)=(1-q_i)\,g_i\!\left(t-o_i-t0_i\right)\1\{t\ge o_i+t0_i\},
\]
\[
S_i(t)=q_i+(1-q_i)\left[1-G_i\!\left(t-o_i-t0_i\right)\right].
\]

\subsection{Expression-level events}
Let $E_j$ denote expression events built from accumulators/pools/guards.

\paragraph{OR / first-of}
If $E=\bigvee_{j=1}^m E_j$ (first completion among independent branches),
\[
f_E(t)=\sum_{j=1}^m f_{E_j}(t)\prod_{k\ne j}S_{E_k}(t),\qquad
S_E(t)=\prod_{j=1}^m S_{E_j}(t).
\]

\paragraph{AND / all-of}
If $E=\bigwedge_{j=1}^m E_j$ (all finished by $t$),
\[
F_E(t)=\prod_{j=1}^m F_{E_j}(t),\qquad
f_E(t)=\sum_{j=1}^m f_{E_j}(t)\prod_{k\ne j}F_{E_k}(t).
\]

\paragraph{Guard}
For $E=\text{guard}(\text{blocker}=B,\text{reference}=R)$,
\[
f_E(t)=f_R(t)\,S_B^{\mathrm{eff}}(t),\qquad
F_E(t)=\int_0^t f_R(u)\,S_B^{\mathrm{eff}}(u)\,du.
\]

\paragraph{$k$-of-$n$ pool}
If $N(t)=\sum_{i=1}^n \1\{T_i\le t\}$, then for the $k$-th order event:
\[
S_{k:n}(t)=\Prob\!\left(N(t)<k\right),\qquad
f_{k:n}(t)=\frac{d}{dt}\left[1-S_{k:n}(t)\right].
\]

\subsection{Current single observed outcome}
With observed $(R=r,\ rt=t)$ and competitor set $C(r)$:
\[
p(r,t)=d_r(t),
\quad
d_r(t)=f_{\nu(r)}(t)\prod_{c\in C(r)}S_{\nu(c)}(t).
\]
With component mixture $m$ and weights $w_m$:
\[
p(r,t)=\sum_m w_m\,d_{r,m}(t).
\]
This is the current one-event-per-trial target.

\section{A. Multiple outcomes}
\subsection{A1. Multiple labels for one latent event}
If one latent event $e$ is mapped to a vector label $y$ via map $h$:
\[
p(Y=y,\ t)=\sum_{e:\ h(e)=y} d_e(t).
\]
This is a latent-label summation only (no extra time convolution).

\subsection{A2. Multiple observed event outcomes within trial}
Suppose trial observation is an ordered sequence
\[
(r_1,t_1),\ldots,(r_K,t_K),\qquad 0<t_1<\cdots<t_K.
\]
Let $\mathcal H_{k-1}$ be the history/forced-state after the first $k-1$ events.
Define conditional event hazard term:
\[
\lambda_{r_k}(t_k\mid \mathcal H_{k-1})
=
f_{r_k}(t_k\mid \mathcal H_{k-1})
\prod_{c\in C_{r_k}(\mathcal H_{k-1})}S_c(t_k\mid \mathcal H_{k-1}).
\]
Then
\[
p(r_{1:K},t_{1:K})
=
\1\{t_1<\cdots<t_K\}
\prod_{k=1}^K \lambda_{r_k}(t_k\mid \mathcal H_{k-1}).
\]

If times are missing/latent:
\[
p(r_{1:K})
=
\int_{0<t_1<\cdots<t_K<\infty}
\prod_{k=1}^K \lambda_{r_k}(t_k\mid \mathcal H_{k-1})\,dt_{1:K}.
\]
If intermediate outcomes are latent too, also sum over latent labels.

\paragraph{Key new object}
The core new integral for multi-outcome likelihood is an ordered-simplex integral over latent event times.

\section{B. Chained races}
\subsection{Two-stage chain}
Let stage 1 output $(Z_1,T_1)$. Stage 2 starts at
\[
B_2=T_1+\delta(Z_1),
\]
and has relative finish $U_2$, absolute finish $T_2=B_2+U_2$.

If both stages observed:
\[
p(z_1,t_1,z_2,t_2)
=
d^{(1)}_{z_1}(t_1)\,
d^{(2)}_{z_2\mid z_1}\!\left(t_2-t_1-\delta(z_1)\right)
\1\{t_2>t_1+\delta(z_1)\}.
\]

If stage 1 is unobserved:
\[
p(z_2,t_2)
=
\sum_{z_1}\int_0^{t_2-\delta(z_1)}
d^{(1)}_{z_1}(u)\,
d^{(2)}_{z_2\mid z_1}\!\left(t_2-u-\delta(z_1)\right)\,du.
\]
This is a sum of convolutions over hidden stage-1 time.

\subsection{Many chained stages}
Use forward recursion (semi-Markov style):
\[
\alpha_s(z,t)
=
\sum_{z'}\int_0^t
\alpha_{s-1}(z',u)\,k_s(z,t\mid z',u)\,du,
\]
where $k_s$ is the stage-$s$ conditional race kernel.

\section{Likelihood framework placement}
Your low-level event mathematics can remain mostly unchanged. The main extension is at trial likelihood composition:
\begin{itemize}
  \item current: one $(R,rt)$ contribution per trial;
  \item extended: event-sequence contribution, with product form when fully observed, and forward integral/sum recursion when partially observed.
\end{itemize}

\section{Computational implications}
\begin{center}
\begin{tabularx}{\textwidth}{@{}lXX@{}}
\toprule
Case & New math cost & Efficient strategy \\
\midrule
All event times observed & Mostly repeated densities/survivals & cache node evals by state/time \\
Hidden event times & ordered simplex integrals & forward recursion + 1D quadrature \\
Hidden chained stage(s) & convolutions & adaptive 1D quadrature or FFT on grid \\
Many latent branches & large sums of tiny terms & log-sum-exp accumulation \\
\bottomrule
\end{tabularx}
\end{center}

\paragraph{Practical recommendation}
For small/moderate latent dimension, adaptive Gauss-Kronrod recursion is natural. For repeated convolutions on fixed grids, FFT methods can reduce cost from $O(N^2)$ to $O(N\log N)$ per convolution.

\section{Takeaway}
Mathematically, both requested extensions require replacing the single-event trial target with a \emph{path probability} over event histories:
\[
\text{(products of conditional hazards)} \times \text{(integrals over latent times)} \times \text{(sums over latent outcomes)}.
\]
Your current event/pool/guard primitives are already close to what this requires; the major change is trial-level composition and inference recursion.

\end{document}
